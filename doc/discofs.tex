% vim: foldmarker=(((,))) ts=2 sw=2

\documentclass[a4paper]{article}
\usepackage[utf8]{inputenc}
\usepackage[english]{babel}
\usepackage[T1]{fontenc}
\usepackage[usenames,dvipsnames,svgnames]{xcolor}
\usepackage[colorlinks=true, linkcolor=black, urlcolor=blue!80!black]{hyperref}
\usepackage{ifthen}
\usepackage{listings}
\lstset{
	language=C,
	stringstyle=\color{purple},
	numbers=left,
	basicstyle=\ttfamily,
	numberstyle=\tiny\ttfamily,
	numbersep=5pt,
	showstringspaces=false,
	captionpos=b,
}
\usepackage{courier}
\usepackage{underscore}
\usepackage{xspace}

% macro definitions (((
\newcommand{\software}[1]{\textsc{\textbf{#1}}\xspace}
\newcommand{\discofs}{\software{DISCOFS}}
\newcommand{\fuse}{\software{FUSE}}
\newcommand{\sqlite}{\software{SQLite}}

\newcommand{\component}[1]{\hyperref[comp:#1]{\textsc{#1}}\xspace}
\newcommand{\job}{\component{job}}
\newcommand{\jobs}{\job{}s\xspace}
\newcommand{\jobtype}[1]{\texttt{#1}\xspace}
\newcommand{\push}{\jobtype{push}}
\newcommand{\pull}{\jobtype{pull}}
\newcommand{\sync}{\component{sync}}
\newcommand{\cache}{\component{cache}}
\newcommand{\remote}{\component{remote}}
\newcommand{\database}{\component{database}}

\newcommand{\state}{\component{state}}
\newcommand{\online}{\component{online}}
\newcommand{\offline}{\component{offline}}

\newcommand{\workerthread}{\hyperref[comp:workerthread]{\textsc{worker thread}}\xspace}
\newcommand{\statecheckthread}{\hyperref[comp:statecheckthread]{\textsc{state check thread}}\xspace}


\newcommand{\loglevel}[1]{\textit{#1}\xspace}
\newcommand{\ERROR}{\loglevel{ERROR}}
\newcommand{\INFO}{\loglevel{INFO}}
\newcommand{\VERBOSE}{\loglevel{VERBOSE}}
\newcommand{\DEBUG}{\loglevel{DEBUG}}
\newcommand{\FSOP}{\loglevel{FSOP}}



\newcommand{\fsopref}[1]{\hyperref[fsop:#1]{\texttt{#1}}}
\newcommand{\sectionref}[1]{\hyperref[#1]{section~\ref{#1}}}
%)))

\title{\discofs \\ \textsc{\small \textbf{DISCO}nnected \textbf{F}ile \textbf{S}ystem} \\ Documentation}
\author{Robin Martinjak}
\date{\today}

\begin{document}

\maketitle
\pagebreak

\tableofcontents
\pagebreak


\section{Introduction} %(((

\subsection{Purpose of \discofs} %(((
The main goal of \discofs is to provide a transparent layer between the local
file system of a computer and a networked file system residing on a remote
server. \discofs is designed to be completely agnostic of the types of
\emph{both} file systems and to work without the need of installing any
additional software on the server side.

The user is able to access all files and directories at all times, even if there
is currently no connection to the server. To accomplish this, all operations are
performed on a local copy of the file system and then replayed onto the remote
side, immediately if the user is ``online'', else after the connection has been
re-established. Changes on the remote file system are detected automatically,
either when the user accesses the file or by periodic scanning.
%)))

\subsection{Example use case} %(((
\newcommand{\somename}{Alice\xspace}
\newcommand{\somecompany}{ACME~Inc.\ }
\newcommand{\sometown}{Atlantis}
\newcommand{\anothertown}{Bikini Bottom}

\somename works at \somecompany in \sometown. Sometimes she has to travel 
to a subsidiary in \anothertown. The files for a project she works on are
stored on a NFS share at the headquarters in \sometown.

While on the train to or from \anothertown, \somename might want to do some work
on the project. So before leaving, she copies the project directory to her
laptops hard drive to have them available during the journey. When she returns
to the \somecompany HQ in \sometown, she has to copy any files she changed back
to the NFS share, of course after confirming that nobody else has touched them.

With \discofs, these steps are performed automatically. \somename simply mounts
a \discofs instance over the NFS share on her laptop. \discofs will constantly
synchronise the data between NFS and her hard drive while she is at work in
\sometown.
When \somename takes her laptop away, she has all files for the project with
her. Any changes she makes will be recorded by \discofs and applied to the NFS
share once she is back in \sometown. If a file \somename modified was also
changed on the NFS share, \discofs will resolve this by keeping one of the two
versions of the file and deleting or renaming the other (see
\sectionref{sec:conflicts} for details on conflict resolution).
%)))

%)))

\section{Components} %(((

\subsection{Overview} %(((
\discofs consists of the following components: 

\begin{itemize}
	\item The functions which implement the file system operations (see
		\hyperref[sec:fsops]{appendix~\ref{sec:fsops}}).
	\item The \cache, a local copy of the remote file system.
	\item The \sync table keeping track of the synchronisation time of the file
		system objects.
	\item The \job system to organize the replication of performed operations
		onto the remote file system.
	\item An \database for persistent storage of \job and \sync data.
	\item A \statecheckthread that tests the availability of the remote file
		system.
	\item A \workerthread that performs \jobs or polls the remote file system
		for changes.
\end{itemize}
%)))

\subsection{The \cache} %(((
\label{comp:cache}
The \cache contains a local copy of the \remote file system. Reading and writing
data is always performed on the \cache. If a file is opened that doesn't exist
in the \cache yet, it is instantly pulled from the \remote file system.
%)))

\subsection{The \sync table} %(((
\label{comp:sync}
\discofs needs to keep track of the time a file was last synchronised. Each time
a file on \remote is changed or modified, the \texttt{mtime} and \texttt{ctime} on
the \remote file are recorded. If the time stamps of a \remote file are newer than
the last recorded ones, \discofs considers the file to be modified/changed. 

The \sync table is implemented as an in-memory hash table to provide fast
lookups. Changes made to it are periodically saved to the \database by the
\workerthread (see \sectionref{comp:workerthread}) and on program termination
(i.e. unmounting \discofs).
%)))

\subsection{The \job system} %(((
\label{comp:job}
\discofs uses a \job system to manage the synchronisation between \cache and
\remote. 
Every operation performed in \online state is performed on the
\remote side immediately, except for \fsopref{write}.

Performing a \fsopref{write} while \online or any operation that
\emph{modifies} the file sytem while \offline is recorded as a \job, along with
their priority (see below) and time stamp. New \jobs are added to a queue and
occasionally saved to the \database.

\subsubsection{High priority jobs}
The following \jobs have \textbf{high} priority, because other \jobs might rely
on the fact that a file or directory does or doesn't exist:
\begin{itemize}
	\item \fsopref{create} and \fsopref{mknod}
	\item \fsopref{mkdir}
	\item \fsopref{unlink}
\end{itemize}

\subsubsection{Medium priority jobs}
The following \jobs have \textbf{medium} priority:
\begin{itemize}
	\item \fsopref{rename}
	\item \fsopref{symlink}
	\item \fsopref{link}
	\item \fsopref{rmdir}
	\item \fsopref{chmod}
	\item \fsopref{chown}
	\item \fsopref{setxattr}
\end{itemize}

\subsubsection{Low priority jobs}
The \jobs for transferring files have \textbf{low} priority because they can't be applied
immediately:
\begin{itemize}
	\item \push: transfer file from \cache to \remote. \push jobs are created if a
			file was written to.
	\item \pull: transfer file from \remote to \cache. \pull jobs are created by
			the \workerthread when polling the \remote file system.
\end{itemize}


\subsubsection{Job selection}
Selecting which job has to be performed next is based on the following criteria:

\begin{enumerate}
		\item Select jobs whose time stamps are in the past.
		\item From these, select jobs which have the highest priority in the queue.
		\item from these, select the one with the time stamp furthest in the past.
\end{enumerate}

The first condition makes it possible to defer jobs to be retried after a
certain amount of time.

\subsubsection{Returning finished or failed jobs}
\label{sec:job_return}
Every job handed out to the \workerthread is returned, with success or failure
indication. Successfully finished jobs are removed from the job queue (and
\database). Failed jobs are requeued, with their time stamp set to a moment in
the future so they can be retried. The number of retries is limited.
%)))

\subsection{The \database} %(((
\label{comp:database}
\discofs stores its metadata, consisting of
\begin{itemize}
	\item the \sync table
	\item the \job queue
	\item information about hard links
	\item a few configuration values
\end{itemize}
in an \sqlite database. This is especially handy for the \job queue because
selecting which job to perform next can be done easily with just one SQL query.
%)))

\subsection{The \statecheckthread} %(((
\label{comp:statecheckthread}
The \statecheckthread periodically tests whether the \remote file system is
available. The \state can be either \online or \offline. The following criteria
must be matched:
\begin{itemize}
	\item The \remote file system must be mounted. This is currently checked by
		comparing the \texttt{st_dev} entry of the \remote mount point an its
		parent directory. See the \texttt{stat(2)} man page for details.

	\item \emph{Optional}: If a hostname or IP address was specified with the
		\texttt{host} mount option, the given host must be reachable via the 
		\texttt{ping(8)} utility.

	\item \emph{Optional}: If a ``PID file'' was specified with the \texttt{pid}
		mount option, a process with the PID contained in this file must exist,
		i.e. \texttt{kill(pid,~0)} returns zero (see \texttt{kill(2)} for
		details).
\end{itemize}
%)))

\subsection{The \workerthread} %(((
\label{comp:workerthread}
Under certain circumstances, other parts of \discofs will \emph{block} the
\workerthread to prevent it from hogging the bandwidth by performing a
\push or \pull job.
The \workerthread performs the following tasks:

\begin{itemize}
	\item Store the \job queue and \sync table to the \database.
	\item If the \state is \offline: sleep a few seconds (or until woken up by
		the \statecheckthread).
	\item If the \state is \online and the \workerthread is not blocked, do one
			of the following tasks, whichever one can be applied.
\end{itemize}


\subsubsection{Transferring files} %(((
\label{sec:worker_transfer}
If a file has to be transferred from or to the \remote file system, this is done
in small parts per iteration of the \workerthread{}s main loop. If there is
currently a transfer job (i.e. \push or \pull) in progress, the
\workerthread will continue. If the transfer is finished or an error is
encountered, the current job is returned to the \job system (see
\sectionref{sec:job_return}).
%)))

\subsubsection{Performing jobs} %(((
\label{sec:worker_perform_job}
If no file transfer is in progress, the \workerthread gets a job from the \job
system and tries to perform it. If it is a \push or \pull job, this will start a
new transfer which will (probably) be resumed in the next iteration as described
in \ref{sec:worker_transfer}. Else (or if the transfer was finished or failed
instantly), the job will is returned to the \job system (see
\sectionref{sec:job_return}).
%)))

\subsubsection{Scanning the \remote file system for changes} %(((
\label{sec:worker_scan_remote}
If there are no transfers to be resumed and no jobs to be performed, the
\workerthread will scan the \remote file system for changes. This is done using
a queue which contains directory names to scan. If the queue is empty, the
\workerthread will sleep for an amount of time (configurable with the
\texttt{scan} mount option) and then start from the root directory. When
scanning a directory, entries that are subdirectories are added to the queue.

For regular files and symbolic links the time stamps will be compared to the
corresponding entries in the \sync table. Symbolic links will be synchronised
instantly. If a regular file is new or modified, the \workerthread will
schedule a \pull job.
%)))

%)))

%)))

\section{Development} %(((

\subsection{Status}
Hard links are currently broken in multiple ways. See the
\href{https://github.com/rmartinjak/discofs/issues}{issues on GitHub}.
%)))

% APPENDIX

\pagebreak
\clearpage
\pagenumbering{Roman}

\begin{appendix} %(((
\section{Supported file system operations} %(((
\label{sec:fsops}
\fuse defines a set of operations (see the
\href{http://fuse.sourceforge.net/doxygen/structfuse__operations.html}{\fuse
API documentation}) of which \discofs implements the following:

\newcommand{\fsop}[3]{ %(((
\item \label{fsop:#1} \texttt{#1}
	\ifthenelse{\equal{#2}{-}}
		{(No corresponding \textsc{POSIX} function)}
		{(\textsc{POSIX}: \texttt{\ifthenelse{\equal{#2}{}}{#1}{#2}})}
		\\
	#3 \\
}
%)))
\begin{itemize}\itemsep0pt
\fsop{statfs}{statvfs}{Get file system statistics.}

\fsop{getattr}{stat/lstat}{Get file status for a path.}
\fsop{fgetattr}{fstat}{Get file status for an open file descriptor.}
\fsop{access}{}{Determine accessibility of a file.}

\fsop{create}{open}{Create and open a file.}
\fsop{open}{}{Open an \emph{existing} file.}
\fsop{truncate}{}{Truncate a file to a specified length.}
\fsop{read}{}{Read data from a file decriptor.}
\fsop{write}{}{Write data to a file descriptor.}
\fsop{release}{-}{Called when \emph{all} references to an open file are closed. \fuse guarantees that there is exactly one \texttt{release()} for every \texttt{open()}.}

\fsop{flush}{-}{Called on each \texttt{close()} of a file descriptor.
There may be multiple \texttt{flush()} calls for an \texttt{open()} call.}

\fsop{fsync}{fsync/fdatasync}{Synchronise file content to the underlying device.}

\fsop{symlink}{}{Create a symbolic link.}
\fsop{readlink}{}{Read the contents of a symbolic link.}

\fsop{mknod}{}{Create a (special) file.}
\fsop{unlink}{}{Delete a name from the file system.}
\fsop{link}{}{Create a new link (hard link) to an existing file.}

\fsop{rename}{}{Rename (and/or move) a file.}

\fsop{chown}{}{Change ownership of a file.}
\fsop{chmod}{}{Change access permission of a file.}
\fsop{utimens}{utimensat}{Change file timestamps (nanosecond precision).}
\fsop{mkdir}{}{Create directory.}
\fsop{rmdir}{}{Remove directory.}
\fsop{opendir}{}{Open directory stream.}
\fsop{readdir}{}{Read from directory stream.}
\fsop{releasedir}{}{Close directory stream.}
\fsop{fsyncdir}{fsync/fdatasync}{Like \fsopref{fsync}, but for directories.}

\fsop{setxattr}{lsetxattr}{Set extended attribute.}
\fsop{getxattr}{lgetxattr}{Retrieve extended attribute.}
\fsop{listxattr}{llistxattr}{List extended attributes.}

\end{itemize}
%)))

\end{appendix}%)))

\end{document}
