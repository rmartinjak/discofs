% vim: foldmarker=(((,))) ts=2 sw=2

\documentclass[a4paper]{article}
\usepackage[utf8]{inputenc}
\usepackage[english]{babel}
\usepackage[T1]{fontenc}
\usepackage[usenames,dvipsnames,svgnames]{xcolor}
\usepackage[colorlinks=true, linkcolor=black, urlcolor=blue!80!black]{hyperref}
\usepackage{ifthen}
\usepackage{listings}
\lstset{
	language=C,
	backgroundcolor=\color{white!90!black},
	stringstyle=\color{purple},
	commentstyle=\color{green!40!black},
	numbers=none,
	basicstyle=\small\ttfamily,
	numbersep=5pt,
	showstringspaces=false,
	captionpos=b,
	breaklines=true,
}
\usepackage{courier}
\usepackage{underscore}
\usepackage{xspace}
\usepackage{multicol}

% macro definitions (((
\newcommand{\myemail}{\href{mailto:rob@rmartinjak.de}{rob@rmartinjak.de}\xspace}
\newcommand{\software}[1]{\textsc{\textbf{#1}}\xspace}
\newcommand{\discofs}{\software{DISCOFS}}
\newcommand{\fuse}{\href{http://fuse.sourceforge.net/}{\software{FUSE}\xspace}}
\newcommand{\git}{\href{http://git-scm.com/}{\software{Git}}\xspace}
\newcommand{\sqlite}{\href{http://sqlite.org/}{\software{SQLite}}\xspace}

\newcommand{\github}{\href{https://github.com}{GitHub}\xspace}

\newcommand{\keyword}[1]{\hyperref[keyword:#1]{\textsc{#1}}\xspace}
\newcommand{\job}{\keyword{job}}
\newcommand{\jobs}{\job{}s\xspace}
\newcommand{\jobtype}[1]{\texttt{#1}\xspace}
\newcommand{\push}{\jobtype{push}}
\newcommand{\pull}{\jobtype{pull}}
\newcommand{\sync}{\keyword{sync}}
\newcommand{\cache}{\keyword{cache}}
\newcommand{\remote}{\keyword{remote}}
\newcommand{\database}{\keyword{database}}

\newcommand{\state}{\keyword{state}}
\newcommand{\online}{\keyword{online}}
\newcommand{\offline}{\keyword{offline}}

\newcommand{\conflict}{\keyword{conflict}}

\newcommand{\workerthread}{\hyperref[keyword:workerthread]{\textsc{worker thread}}\xspace}
\newcommand{\statecheckthread}{\hyperref[keyword:statecheckthread]{\textsc{state check thread}}\xspace}


\newcommand{\loglevel}[1]{\textit{#1}\xspace}
\newcommand{\ERROR}{\loglevel{ERROR}}
\newcommand{\INFO}{\loglevel{INFO}}
\newcommand{\VERBOSE}{\loglevel{VERBOSE}}
\newcommand{\DEBUG}{\loglevel{DEBUG}}
\newcommand{\FSOP}{\loglevel{FSOP}}



\newcommand{\fsopref}[1]{\hyperref[fsop:#1]{\texttt{#1}}}
\newcommand{\sectionref}[1]{\hyperref[#1]{section~\ref{#1}}}
%)))

\title{\discofs \\ \textsc{\small \textbf{DISCO}nnected \textbf{F}ile \textbf{S}ystem} \\ Documentation}
\author{Robin Martinjak}
\date{\today}

\begin{document}

\maketitle
\pagebreak

\tableofcontents
\pagebreak


\section{Introduction} %(((

\subsection{Purpose of \discofs} %(((
The main goal of \discofs is to provide a transparent layer between the local
file system of a computer and a networked file system residing on a remote
server. \discofs is designed to be completely agnostic of the types of
\emph{both} file systems and to work without the need of installing any
additional software on the server side.

The user is able to access all files and directories at all times, even if there
is currently no connection to the server. To accomplish this, all operations are
performed on a local copy of the file system and then replayed onto the remote
side, immediately if the user is ``online'', else after the connection has been
re-established. Changes on the remote file system are detected automatically,
either when the user accesses the file or by periodic scanning.
%)))

\subsection{Example use case} %(((
\newcommand{\somename}{Alice\xspace}
\newcommand{\somecompany}{ACME~Inc.\ }
\newcommand{\sometown}{Atlantis}
\newcommand{\anothertown}{Bikini Bottom}

\somename works at \somecompany in \sometown. Sometimes she has to travel
to a subsidiary in \anothertown. The files for a project she works on are
stored on a NFS share at the headquarters in \sometown.

While on the train to or from \anothertown, \somename might want to do some work
on the project. So before leaving, she copies the project directory to her
laptops hard drive to have them available during the journey. When she returns
to the \somecompany HQ in \sometown, she has to copy any files she changed back
to the NFS share, of course after confirming that nobody else has touched them.

With \discofs, these steps are performed automatically. \somename simply mounts
a \discofs instance over the NFS share on her laptop. \discofs will constantly
synchronise the data between NFS and her hard drive while she is at work in
\sometown.
When \somename takes her laptop away, she has all files for the project with
her. Any changes she makes will be recorded by \discofs and applied to the NFS
share once she is back in \sometown. If a file \somename modified was also
changed on the NFS share, \discofs will resolve this by keeping one of the two
versions of the file and deleting or renaming the other (see
\sectionref{keyword:conflict} for details on conflict resolution).
%)))

\subsection{Installing and running \discofs} %(((

To download, install and use \discofs, the following tools and libraries are
needed:

\begin{itemize}
	\item A \git client to obtain the source code from \github (optional but
		recommended).
	\item A C compiler (preferably
		\href{http://gnu.org/software/gcc}{\software{GCC}}
		or
		\href{http://clang.llvm.org/}{\software{clang}}).
	\item \href{http://gnu.org/software/make}{\software{GNU Make}} or a
		similar Make implementation.
	\item The \fuse library and headers.
	\item The \sqlite library and headers.
\end{itemize}


Installing \discofs is done by performing the following steps:

\lstset{language=bash}
\begin{lstlisting}[deletekeywords={cd},]
# clone the git repository:
git clone --recursive https://github.com/rmartinjak/discofs.git
cd discofs

# Run the configure script.
# By default, the executable and manual will be installed
# in /usr/local. To change this, use the --prefix
# argument, e.g.:
#./configure --prefix=/usr
./configure

# Run Make
make

# Install the executable and manual.
# Depending on the prefix you might need to do this
# using a privileged user acoount.
make install
\end{lstlisting}

To mount \discofs, run the \lstinline|discofs| command. Assuming the networked
file system you want to mirror is mounted at \lstinline|/home/alice/nfs| and
your desired mount point is \lstinline|/home/alice/discofs|, run
\begin{lstlisting}
discofs /home/alice/nfs /home/alice/discofs
\end{lstlisting}

The available command line arguments and mount options are documented in the
\discofs manual. To access it, type \lstinline|man discofs|.

\lstset{language=C}
%)))

%)))


\section{Components} %(((

\subsection{Overview} %(((
\discofs consists of the following components:

\begin{itemize}
	\item The functions which implement the file system operations (see
		\hyperref[appendix:fsops]{appendix~\ref{appendix:fsops}}).
	\item The \cache, a local copy of the remote file system.
	\item The \sync table keeping track of the synchronisation time of the file
		system objects.
	\item The \job system to organise the replication of performed operations
		onto the remote file system.
	\item An \database for persistent storage of \job and \sync data.
	\item A \statecheckthread that tests the availability of the remote file
		system.
	\item A \workerthread that performs \jobs or polls the remote file system
		for changes.
\end{itemize}
%)))

\subsection{The \cache} %(((
\label{keyword:cache}
The \cache contains a local copy of the \remote file system. Reading and writing
data is always performed on the \cache. If a file is opened that doesn't exist
in the \cache yet, it is instantly pulled from the \remote file system.
%)))

\subsection{The \sync table} %(((
\label{keyword:sync}
\discofs needs to keep track of the time a file was last synchronised. Each time
a file on \remote is changed or modified, the \texttt{mtime} and \texttt{ctime} on
the \remote file are recorded. If the time stamps of a \remote file are newer than
the last recorded ones, \discofs considers the file to be modified/changed.

The \sync table is implemented as an in-memory hash table to provide fast
lookups. The key used is the file path \emph{relative} to the mount point.
Changes made to it are periodically saved to the \database by the
\workerthread (see \sectionref{keyword:workerthread}) and on program termination
(i.e. unmounting \discofs).
%)))

\subsection{The \job system} %(((
\label{keyword:job}
\discofs uses a \job system to manage the synchronisation between \cache and
\remote.
Every operation performed in \online state is performed on the
\remote side immediately, except for \fsopref{write}.

Performing a \fsopref{write} while \online or any operation that
\emph{modifies} the file sytem while \offline is recorded as a job, along with
their priority (see below) and time stamp. New jobs are added to a queue and
occasionally saved to the \database.

\subsubsection{Job priorities}
Below is an overview of all job types. A few jobs have \textbf{high}
priority, because other jobs might rely on the fact that a file or directory
does or doesn't exist. The
\texttt{push}\footnote{transfer file from \cache to \remote. \push jobs are created if a
	file was written to.}
and
\texttt{pull}\footnote{transfer file from \remote to \cache. \pull jobs are created by
	the \workerthread when polling the \remote file system.}
job types for transferring files have \textbf{low} priority because they can't be applied
immediately.

\begin{multicols}{3}

\begin{tabular}{l}
	\textbf{High priority} \\
	\fsopref{create} / \fsopref{mknod} \\
	\fsopref{mkdir} \\
	\fsopref{unlink} \\
\end{tabular}

\vfill
\columnbreak
\begin{tabular}{l}
	\textbf{Medium priority} \\
	\fsopref{rename} \\
	\fsopref{symlink} \\
	\fsopref{link} \\
	\fsopref{rmdir} \\
	\fsopref{chmod} \\
	\fsopref{chown} \\
	\fsopref{setxattr} \\
\end{tabular}

\vfill
\columnbreak
\begin{tabular}{l}
\textbf{Low priority} \\
	\push \\
	\pull \\
\end{tabular}
\end{multicols}


\subsubsection{Job selection}
Selecting which job has to be performed next is based on the following criteria:

\begin{enumerate}
		\item Select jobs whose time stamps are in the past.
		\item From these, select jobs which have the highest priority in the queue.
		\item From these, select the one with earliest time stamp.
\end{enumerate}

The first condition makes it possible to defer jobs to be retried after a
certain amount of time.

\subsubsection{Returning finished or failed jobs}
\label{sec:job_return}
Every job handed out to the \workerthread is returned, with success or failure
indication. Successfully finished jobs are removed from the job queue (and
\database). Failed jobs are requeued, with their time stamp set to a moment in
the future so they can be retried. The number of retries is limited.
%)))

\subsection{The \database} %(((
\label{keyword:database}
\discofs stores its metadata, consisting of
\begin{itemize}
	\item the \sync table
	\item the \job queue
	\item information about hard links
	\item a few configuration values
\end{itemize}
in an \sqlite database. This is especially handy for the \job queue because
selecting which job to perform next can be done easily with just one SQL query.
%)))

\subsection{The \statecheckthread} %(((
\label{keyword:state}
\label{keyword:statecheckthread}
The \statecheckthread periodically tests whether the \remote file system is
available. The \state can be either \online or \offline. The following criteria
must be matched:
\begin{itemize}
	\item The \remote file system must be mounted. This is currently checked by
		comparing the \texttt{st_dev} entry of the \remote mount point an its
		parent directory. See the \texttt{stat(2)} man page for details.

	\item \emph{Optional}: If a hostname or IP address was specified with the
		\texttt{host} mount option, the given host must be reachable via the
		\texttt{ping(8)} utility.

	\item \emph{Optional}: If a ``PID file'' was specified with the \texttt{pid}
		mount option, a process with the PID contained in this file must exist,
		i.e. \texttt{kill(pid,~0)} returns zero (see \texttt{kill(2)} for
		details).
\end{itemize}

If the \state changes from \offline to \online, the sleeping \workerthread is
``woken up'' from sleeping.
%)))

\subsection{The \workerthread} %(((
\label{keyword:workerthread}
%(((
Under certain circumstances, other parts of \discofs will \emph{block} the
\workerthread to prevent it from hogging the bandwidth by performing a
\push or \pull job.
The \workerthread performs the following tasks:

\begin{itemize}
	\item Store the \job queue and \sync table to the \database.
	\item If the \state is \offline: sleep a few seconds (or until woken up by
		the \statecheckthread).
	\item If the \state is \online and the \workerthread is not blocked, do one
			of the following tasks, whichever one can be applied.
\end{itemize}
%)))

\subsubsection{Transferring files} %(((
\label{sec:worker_transfer}
If a file has to be transferred from or to the \remote file system, this is done
in small parts per iteration of the \workerthread{}s main loop. If there is
currently a transfer job (i.e. \push or \pull) in progress, the
\workerthread will continue. If the transfer is finished or an error is
encountered, the current job is returned to the \job system (see
\sectionref{sec:job_return}).
%)))

\subsubsection{Performing jobs} %(((
\label{sec:worker_perform_job}
If no file transfer is in progress, the \workerthread gets a job from the \job
system and tries to perform it. If it is a \push or \pull job, this will start a
new transfer which will (probably) be resumed in the next iteration as described
in \ref{sec:worker_transfer}. Else (or if the transfer was finished or failed
instantly), the job will is returned to the \job system (see
\sectionref{sec:job_return}).
%)))

\subsubsection{Scanning the \remote file system for changes} %(((
\label{sec:worker_scan_remote}
If there are no transfers to be resumed and no jobs to be performed, the
\workerthread will scan the \remote file system for changes. This is done using
a queue which contains directory names to scan. If the queue is empty, the
\workerthread will sleep for an amount of time (configurable with the
\texttt{scan} mount option) and then start from the root directory. When
scanning a directory, entries that are subdirectories are added to the queue.

For regular files and symbolic links the time stamps will be compared to the
corresponding entries in the \sync table. Symbolic links will be synchronised
instantly. If a regular file is new or modified, the \workerthread will
schedule a \pull job.
%)))
%)))

%)))


\section{File system operations} %(((
This section describes the ``work flow'' of \discofs when performing certain
file system operations.

\subsection{Trivial operations} %(((
Many operations are trivial and need not be explained here. They are generally
just ``handed down'' to the file in the \cache and/or on the \remote file
system. \fuse makes sure that symbolic links are already dereferenced when
needed, that means \discofs will never do so (e.g. use \texttt{lstat} instead of
\texttt{stat}).

For a list of all implemented operations, see
\hyperref[appendix:fsops]{appendix~\ref{appendix:fsops}}.

%)))

\subsection{Get file attributes: \fsopref{getattr}} %(((
This operation is similar to \texttt{stat(2)}. The implementation is quite
trivial, but \fsopref{getattr} deserves mentioning because it is the most
frequently called operation. \discofs implements \fsopref{getattr} by calling
\texttt{lstat} on the file in the \cache. If this fails because the file does
not yet exist and the status is \online, \discofs will try a \texttt{lstat} on the
\remote file.
%)))

\subsection{Read directory contents: \fsopref{readdir}} %(((
When reading directory contents while \online, \discofs will read both the
\cache and \remote directory, and merge the contents (removing duplicates) into
a single list.
%)))

\subsection{Rename a file: \fsopref{rename}} %(((
Renaming is a quite expensive operation in \discofs, especially for directories.
\discofs has to update the \sync table, \job queue and \database entries for the
matching path. In case of a directory, the path for any object below it must be
updated.

During re-synchronisation, attempting a \fsopref{rename} on a modified
\remote \emph{target} will result in a \conflict.
%)))

\subsection{Open a file: \fsopref{open} / \fsopref{create}} %(((
Opening a file will always be performed in the \cache. If the target doesn't yet
exist in the \cache or has been modified on the \remote file system, it is
pulled \emph{instantly} before opening. This way the user will never work with
an outdated version of the file.
As long as a file is opened, there will be no attempts to transfer it (in
either direction).
%)))

\subsection{Modify file content: \fsopref{write} / \fsopref{truncate}} %(((
Performing a \fsopref{write} on a file will \emph{mark} it as ``dirty''. When
all file descriptors of the file are closed (i.e. upon \fsopref{release}), a
\texttt{push} job will be scheduled.
The \fsopref{truncate} operation behaves similar when \offline. When \online,
it will be applied to the \remote file immediately.

Performing a \texttt{push} job on a file that has been modified on the \remote
file system will result in a \conflict.
%)))

\subsection{Remove a file: \fsopref{unlink}} %(((
When a file is removed, its entry in the \sync table and all outstanding jobs
for the file are deleted. This includes \fsopref{rename} jobs with the file path
as the target name. If the file is currently being transferred (either
direction), this transfer will be aborted.
If an \fsopref{unlink} job is attempted on a modified file, this will \emph{NOT}
cause a \conflict but simply trigger scheduling of a \texttt{pull} job.
%)))

% hard links don't work properly yet
% \subsection{Create a hard link: \fsopref{link}} %(((
%)))

\subsection{Conflicts} %(((
\label{keyword:conflict}
A \conflict is a situation in where a file has been modified both in the \cache
and on the \remote file system independently. \discofs will not attempt to merge
those modifications. Instead, it will decide which file to keep according to a
user-specified (with the \texttt{conflict} mount option) rule:
\begin{itemize}
	\item \textbf{newer}: Keep the file with the latest time stamp.
	\item \textbf{mine}: Always keep the file from the \cache.
	\item \textbf{theirs}: Always keep the file from the \remote file system.
\end{itemize}

The other file will be \emph{deleted} or \emph{renamed} if at least one of the
\texttt{bprefix} or \texttt{bsuffix} mount options is specified (see
\lstinline|man discofs| for details).

%)))

%)))


\section{Development} %(((

\subsection{Status}
Most features needed for everyday usage are implemented and seem to work pretty
well. More extensive testing is needed to find corner cases that produce
problematic behaviour. An exception to this are hard links, they are currently
broken in multiple ways.

For a list of what else is wrong, see the
\href{https://github.com/rmartinjak/discofs/issues}{issues on GitHub}.

\subsection{How to contribute}
Use it! If something strange happens, try to reproduce it and submit a bug
report, either as an issue on \github or to \myemail.
Pull requests are of course also welcome!

\subsection{Future plans}
The ultimate goal is making \discofs reliable enough to mount a users
\texttt{\$HOME} directory. If that is given, performance should be improved,
especially renaming files is currently very expensive.
Apart from that, no additional features are planned yet.
%)))


% APPENDIX

\pagebreak
\clearpage
\pagenumbering{Roman}

\begin{appendix} %(((
\section{Supported file system operations} %(((
\label{appendix:fsops}
\fuse defines a set of operations (see the
\href{http://fuse.sourceforge.net/doxygen/structfuse__operations.html}{\fuse
API documentation}) of which \discofs implements the following:

\newcommand{\fsop}[3]{ %(((
\item \label{fsop:#1} \texttt{#1}
	\ifthenelse{\equal{#2}{-}}
		{(No corresponding \textsc{POSIX} function)}
		{(\textsc{POSIX}: \texttt{\ifthenelse{\equal{#2}{}}{#1}{#2}})}
		\\
	#3 \\
}
%)))
\begin{itemize}\itemsep0pt
\fsop{statfs}{statvfs}{Get file system statistics.}

\fsop{getattr}{stat/lstat}{Get file status for a path.}
\fsop{fgetattr}{fstat}{Get file status for an open file descriptor.}
\fsop{access}{}{Determine accessibility of a file.}

\fsop{create}{open}{Create and open a file.}
\fsop{open}{}{Open an \emph{existing} file.}
\fsop{truncate}{}{Truncate a file to a specified length.}
\fsop{read}{}{Read data from a file decriptor.}
\fsop{write}{}{Write data to a file descriptor.}
\fsop{release}{-}{Called when \emph{all} references to an open file are closed. \fuse guarantees that there is exactly one \texttt{release()} for every \texttt{open()}.}

\fsop{flush}{-}{Called on each \texttt{close()} of a file descriptor.
There may be multiple \texttt{flush()} calls for an \texttt{open()} call.}

\fsop{fsync}{fsync/fdatasync}{Synchronise file content to the underlying device.}

\fsop{symlink}{}{Create a symbolic link.}
\fsop{readlink}{}{Read the contents of a symbolic link.}

\fsop{mknod}{}{Create a (special) file.}
\fsop{unlink}{}{Delete a name from the file system.}
\fsop{link}{}{Create a new link (hard link) to an existing file.}

\fsop{rename}{}{Rename (and/or move) a file.}

\fsop{chown}{}{Change ownership of a file.}
\fsop{chmod}{}{Change access permission of a file.}
\fsop{utimens}{utimensat}{Change file timestamps (nanosecond precision).}
\fsop{mkdir}{}{Create directory.}
\fsop{rmdir}{}{Remove directory.}
\fsop{opendir}{}{Open directory stream.}
\fsop{readdir}{}{Read from directory stream.}
\fsop{releasedir}{}{Close directory stream.}
\fsop{fsyncdir}{fsync/fdatasync}{Like \fsopref{fsync}, but for directories.}

\fsop{setxattr}{lsetxattr}{Set extended attribute.}
\fsop{getxattr}{lgetxattr}{Retrieve extended attribute.}
\fsop{listxattr}{llistxattr}{List extended attributes.}

\end{itemize}
%)))

\end{appendix}%)))

\end{document}
